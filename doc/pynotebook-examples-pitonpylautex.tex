% !TeX TXS-program:compile = txs:///arara
% arara: lualatex: {shell: yes, synctex: yes, interaction: batchmode}
% arara: lualatex: {shell: yes, synctex: yes, interaction: batchmode} if found('log', '(undefined references|Please rerun|Rerun to get)')

\documentclass[a4paper,11pt]{article}
\def\PLver{0.1.0}
\usepackage[margin=1.5cm]{geometry}
\usepackage{pynotebook}
\usepackage[executable=python]{pyluatex}
\usepackage{codehigh}

\begin{document}

\part*{pynotebook (\PLver), with piton and pyluatex}

\section{Preamble}

\begin{codehigh}
\documentclass{article}
\usepackage{pynotebook}
\usepackage[executable=python]{pyluatex}  % with a specific compilation !!
\end{codehigh}

\section{Examples of text blocks}

\begin{codehigh}
\begin{NotebookPitonMarkdown}{\linewidth}
{\Large\bfseries This is a test for a \textsf{Markdown} block.}

It's possible to use \LaTeX{} formulas, like %
\[
  \left\lbrace\begin{array}{l}
    F_0 = 0\\
    F_1 = 1 \\
    F_{n+2} = F_{n+1} + F_n
  \end{array}\right.
\]
\end{NotebookPitonMarkdown}

\begin{NotebookPitonRaw}{\linewidth}
This is a sample block, with RAW output.

Just to use all capacities of Jupyter notebook ;-)
\end{NotebookPitonRaw}
\end{codehigh}

\begin{NotebookPitonMarkdown}{\linewidth}
{\Large\bfseries This is a test for a \textsf{Markdown} block.}

It's possible to use \LaTeX{} formulas, like %
\[
\left\lbrace\begin{array}{l}
F_0 = 0\\
F_1 = 1 \\
F_{n+2} = F_{n+1} + F_n
\end{array}\right.
\]
\end{NotebookPitonMarkdown}

\begin{NotebookPitonRaw}{\linewidth}
This is a sample block, with RAW output.

Just to use all capacities of Jupyter notebook ;-)
\end{NotebookPitonRaw}

\section{Examples of code blocks (with execution of code !)}

\begin{codehigh}
\begin{NotebookPitonIn}{0.75\linewidth}
def fibonacci_of(n) :
  if n in {0,1} :
    return n
  return fibonacci_of(n-1) + fibonacci_of(n-2)

[fibonacci_of(n) for n in range(15)]
\end{NotebookPitonIn}
\end{codehigh}

\begin{NotebookPitonIn}{0.75\linewidth}
def fibonacci_of(n) :
	if n in {0,1} :
		return n
	return fibonacci_of(n-1) + fibonacci_of(n-2)

[fibonacci_of(n) for n in range(15)]
\end{NotebookPitonIn}

\begin{codehigh}
\begin{NotebookPitonOut}{0.75\linewidth}
def fibonacci_of(n) :
  if n in {0,1} :
    return n
  return fibonacci_of(n-1) + fibonacci_of(n-2)

print([fibonacci_of(n) for n in range(15)])
\end{NotebookPitonOut}
\end{codehigh}

\begin{NotebookPitonOut}{\linewidth}
def fibonacci_of(n) :
	if n in {0,1} :
		return n
	return fibonacci_of(n-1) + fibonacci_of(n-2)

print([fibonacci_of(n) for n in range(15)])
\end{NotebookPitonOut}

\begin{codehigh}
\SetJupyterLng{fr}
\SetJupyterParSkip{\baselineskip}
\setcounter{JupyterIn}{11}
\end{codehigh}

\SetJupyterLng{fr}
\SetJupyterParSkip{\baselineskip}
\setcounter{JupyterIn}{14}

\begin{codehigh}
\begin{NotebookPitonIn}[center]{0.75\linewidth}
def fibonacci_of(n) :
  if n in {0,1} :
    return n
  return fibonacci_of(n-1) + fibonacci_of(n-2)

[fibonacci_of(n) for n in range(15)]
\end{NotebookPitonIn}
\end{codehigh}

\begin{NotebookPitonIn}[center]{0.75\linewidth}
def fibonacci_of(n) :
	if n in {0,1} :
		return n
	return fibonacci_of(n-1) + fibonacci_of(n-2)

[fibonacci_of(n) for n in range(15)]
\end{NotebookPitonIn}

\begin{codehigh}
\begin{NotebookPitonOut}[center]{0.75\linewidth}
def fibonacci_of(n) :
  if n in {0,1} :
    return n
  return fibonacci_of(n-1) + fibonacci_of(n-2)

print([fibonacci_of(n) for n in range(15)])
\end{NotebookPitonOut}
\end{codehigh}

\begin{NotebookPitonOut}[center]{0.75\linewidth}
def fibonacci_of(n) :
	if n in {0,1} :
		return n
	return fibonacci_of(n-1) + fibonacci_of(n-2)

print([fibonacci_of(n) for n in range(15)])
\end{NotebookPitonOut}

\begin{codehigh}
\begin{NotebookPitonConsole}[center]{0.75\linewidth}
def fibonacci_of(n) :
  if n in {0,1} :
    return n
  return fibonacci_of(n-1) + fibonacci_of(n-2)

print([fibonacci_of(n) for n in range(15)])
\end{NotebookPitonConsole}
\end{codehigh}

\begin{NotebookPitonConsole}[center]{0.75\linewidth}
def fibonacci_of(n) :
	if n in {0,1} :
		return n
	return fibonacci_of(n-1) + fibonacci_of(n-2)

print([fibonacci_of(n) for n in range(15)])
\end{NotebookPitonConsole}

\pagebreak

\section{Global example}

\SetJupyterParSkip{default}
\setcounter{JupyterIn}{0}

\begin{NotebookPitonMarkdown}{\linewidth}
{\Large\bfseries This is a test for a \textsf{Markdown} block.}

It's possible to use \LaTeX{} formulas, like %
\[
\left\lbrace\begin{array}{l}
F_0 = 0\\
F_1 = 1 \\
F_{n+2} = F_{n+1} + F_n
\end{array}\right.
\]
\end{NotebookPitonMarkdown}

\begin{NotebookPitonRaw}{\linewidth}
This is a sample block, with RAW output.

Just to use all capacities of Jupyter notebook ;-)
\end{NotebookPitonRaw}

\begin{NotebookPitonIn}{\linewidth}
def fibonacci_of(n) :
	if n in {0,1} :
		return n
	return fibonacci_of(n-1) + fibonacci_of(n-2)

[fibonacci_of(n) for n in range(15)]
\end{NotebookPitonIn}

\begin{NotebookPitonOut}{\linewidth}
def fibonacci_of(n) :
	if n in {0,1} :
		return n
	return fibonacci_of(n-1) + fibonacci_of(n-2)

print([fibonacci_of(n) for n in range(15)])
\end{NotebookPitonOut}

\begin{NotebookPitonRaw}{\linewidth}
Let’s compute Fibonacci terms from 10th to 20th :-)
\end{NotebookPitonRaw}

\begin{NotebookPitonIn}{\linewidth}
[fibonacci_of(n) for n in range(10,21)]
\end{NotebookPitonIn}

\begin{NotebookPitonConsole}{\linewidth}
def fibonacci_of(n) :
	if n in {0,1} :
		return n
	return fibonacci_of(n-1) + fibonacci_of(n-2)

print([fibonacci_of(n) for n in range(10,21)])
\end{NotebookPitonConsole}

\end{document}